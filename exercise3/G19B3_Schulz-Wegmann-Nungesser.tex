\documentclass[ngerman]{gdb-aufgabenblatt}

\renewcommand{\Aufgabenblatt}{3}
\renewcommand{\Ausgabedatum}{Mi. 16.11.2013}
\renewcommand{\Abgabedatum}{Fr. 1.12.2013}
\renewcommand{\Gruppe}{Schulz, Wegmann, Nungesser}


\begin{document}


\section{Konzeptioneller Entwurf}

\begin{tikzpicture}

\node[entity] (e1) {Bio-Molek�l};
\node[attribut] (e1-a1) [above right=-2.5mm and 5mm of e1] {\uline{Molek�lId}} 
    edge (e1); 
\node[attribut] (e1-a2) [below right=-2.5mm and 5mm of e1] {Beschreibung} 
    edge (e1);

\node[entity] (e2) [below left=1cm and 2cm of e1.south] {DNA-Molek�l} 
    edge[erbt] (e1);
\node[attribut] (e2-a1)  [above=5mm of e2] {Nukleotidsequenz} edge (e2); 
\node[attribut] (e2-a3)  [left=5mm of e2] {ChromosonNr} edge (e2); 
\node[attribut] (e2-a2)  [above left=5mm of e2] {StrangOrientierung} edge (e2); 
     
\node[entity] (e3) [below right=1cm and 2cm of e1.south] {mRNA-Molek�l} 
    edge[erbt] (e1); 
\node[attribut] (e3-a1)  [above right=5mm of e3] {Nukleotidsequenz} edge (e3); 
\node[attribut] (e3-a2)  [right=5mm of e3] {ViennaString} edge (e3); 
\node[relationship] (r4) [below=8mm of e1] {transkribiert};
\node[attribut] (r4-a1)  [below left=5mm and -2mm of r4] {StartPosition} 
    edge (r4); 
\node[attribut] (r4-a2)  [below right=5mm and -2mm of r4] {EndPosition}
    edge (r4); 
\path (r4) edge node[at end,anchor=north west] {$n$} (e2);
\path (r4) edge node[at end,anchor=north east] {$1$} (e3);

\node[weakentity] (e5) [below=3cm of e3] {Protein};
\node[attribut] (e5-a1)  [above right=5mm of e5] {Aminos�uresequenz}
edge (e5); 
 \node[attribut] (e5-a2)  [right=5mm of e5] {CATHKlassifikation} edge
 (e5); 
 \node[attribut] (e5-a3)  [below right=5mm of e5] {Molekulargewicht} edge (e5);
\node[weakrelationship] (r2) [below=1cm of e3] {synthetisiert};
\path (r2) edge node[at end,anchor=north west] {$[0;1]$} (e3);
\path (r2) edge node[at end,anchor=south west] {$[1;1]$} (e5);


\node[entity] (e8) [below=3cm of e5]{Dom�ne};
\node[attribut] (e8-a1)  [above left=5mm of e8] {\uline{Dom�nenID}} edge (e8);
\node[attribut] (e8-a2)  [left=5mm of e8] {HiddenMarkovModel} edge (e8);
\node[attribut] (e8-a3)  [below left=5mm of e8] {Funktion} edge (e8);
\node[relationship] (r7) [below=1cm of e5] {enth�lt};
\path (r7) edge node[at end,anchor=north west] {$[0;\infty]$} (e5);
\path (r7) edge node[at end,anchor=south west] {$[1;\infty]$} (e8);

\node[entity] (e6) [above right=2.5cm and 5mm of e1.north]{Artikel};
\node[attribut] (e6-a1)  [above=5mm of e6] {\uline{Titel}} edge (e6);
\node[attribut] (e6-a2)  [above right=5mm of e6] {Datum} edge (e6);
\node[relationship] (r4) [below=5mm of e6] {ver�ffentlicht};
\path (r4) edge node[at end,anchor=north west] {n} (e6);
\path (r4) edge node[at end,anchor=south west] {1} (e1);

\node[entity] (e4) [right=4cm of e6] {Wissenschaftler};
\node[attribut] (e4-a1)  [above=5mm of e4] {\uline{Name}} edge (e4);
\node[multivalentattribut] (e4-a2)  [below=5mm of e4] {Kontaktinformationen} edge (e4);
\node[attribut] (e4-a2-a1)  [below left=5mm of e4-a2] {Telefonnummer} edge
(e4-a2); 
\node[attribut] (e4-a2-a1)  [below=5mm of e4-a2] {E-Mail-Adresse} edge
(e4-a2); \node[relationship] (r1) [left=1cm of e4] {schreibt};
\path (r1) edge node[at end,anchor=south east] {$[0;\infty]$} (e4);
\path (r1) edge node[at end,anchor=south west] {$[1;\infty]$} (e6);

\node[entity] (e7) [above left=2.5cm and 5mm of e1.north] {Organismus};
\node[attribut] (e7-a1)  [above=5mm of e7] {\uline{TaxanomieID}} edge (e7);
\node[attribut] (e7-a2)  [above left=5mm and 5mm of e7] {Name}
    edge (e7); 
 \node[attribut] (e7-a3)  [left=5mm of e7]
    {Trivialname} edge (e7);
\node[relationship] (r3) [below=5mm of e7] {vorkommen};
\path (r3) edge node[at end,anchor=north west] {$[1;\infty]$} (e7);
\path (r3) edge node[at end,anchor=south east] {$[0;\infty]$} (e1);




\end{tikzpicture}


\section{Logischer Entwurf}
\textcolor{red}{TODO}


\section{Relationale Algebra und SQL}

\begin{enumerate}
    \item 
        \begin{enumerate}
            \item 
                \textcolor{red}{TODO}
            \item 
                \textcolor{red}{TODO}
            \item 
                \textcolor{red}{TODO}
            
        \end{enumerate}
    \item 
        \begin{enumerate}
            \item 
                \textcolor{red}{TODO}
            \item
                \textcolor{red}{TODO}
            \item
                \textcolor{red}{TODO}
            \item
                \textcolor{red}{TODO}
                
        \end{enumerate}    

    \item 
        \begin{enumerate}
            \item 
                \textcolor{red}{TODO}
            \item
                 \textcolor{red}{TODO}
                
        \end{enumerate}    

\section{Algebraische Optimierung}

\begin{enumerate}
    \item
        \textcolor{red}{TODO}
    \item  
        Nach der Optimierungsheuristik I "Führe Selektion so früh wie möglich aus!" besitzt der Operatorbaum von b) einen höheren Optimierungsgrad, da die Selektionen in b) wesentlich früher durchgeführt werden.
        Die Relationen in b),welche nach der Selektion, über das Join zusammengeführt werden, sind wesentlich kleiner als die Relationen die bei a) gejoint werden.

\end{enumerate}

\end{document}
